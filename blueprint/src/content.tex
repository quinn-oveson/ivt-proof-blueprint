% In this file you should put the actual content of the blueprint.
% It will be used both by the web and the print version.
% It should *not* include the \begin{document}
%
% If you want to split the blueprint content into several files then
% the current file can be a simple sequence of \input. Otherwise It
% can start with a \section or \chapter for instance.



\section{Definitions}

\begin{definition}[Sequence]
  \label{def:sequence}
  A sequence, denoted $a$ or $\{a_n\}$, is a function $a : ℕ → \mathbb{R}$.
\end{definition}

\begin{definition}[Convergence]
  \label{def:convergence}
  \lean{convergesTo, converges}
  \leanok

  A sequence $\{a_n\}$ converges to $L ∈ \mathbb{R}$ if for all $\varepsilon > 0$ there exists
  $N ∈ ℕ$ such that for all $n ≥ N$, $|a_n - L| < \varepsilon$.
  We say $\{a_n\}$ converges if there exists $L ∈ \mathbb{R}$ such that $\{a_n\}$ converges to $L$.

\end{definition}


\section{Theorems}


\begin{theorem}[Limit Laws]
  \label{thm:limit_laws}

  Let $C \in \mathbb{R}$. Suppose $\{a_n\}$ converges to $L$ and $\{b_n\}$ converges to $K$. Then \\ \\
  (i) $\{C a_n\}$ converges to $C L$ \\
  (ii) $\{a_n + b_n\}$ converges to $L + K$ 

\end{theorem}

\begin{theorem}[Smale 1958]
  \label{thm:sphere_eversion}
  \lean{sphere_eversion}
  \leanok
  \uses{def:immersion}
  There is a homotopy of immersions of $𝕊^2$ into $ℝ^3$ from the inclusion map to
  the antipodal map $a : q ↦ -q$.
\end{theorem}
  
\begin{proof}
  \leanok
  \uses{thm:open_ample, lem:open_ample_immersion}
  This obviously follows from what we did so far. Testing changing my branch
\end{proof}